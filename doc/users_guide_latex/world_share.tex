\chapter{Share your world with other \textsc{Becca} users}
\label{world_share}

\textsc{Becca} was created so that it would be easy to use and share what you create with others. This section describes how to get your world out and get the word out.

\section{Where do I put my world module so that others  can find it?}

The most convenient place to share \textsc{Becca} content is GitHub. Open a GitHub account if you don't have one already, create a repository, and populate it with your world's code. 

The official \textsc{Becca} core code base is hosted on GitHub in one of Matt Chapman's repositories:

\texttt{https://github.com/matt2000/becca}

It contains a tagged version of this code, plus all the incremental commits that are working toward the next version. The core contains the full agent, plus a battery of worlds for benchmarking, and a couple of utility modules. The plan is to keep the \textsc{Becca} core small and trim with only the necessaries for getting a new user up and running. 

Contributed code and modifications will be available separately. The details for how this will happen are still taking shape as our community is still young and small, but for now contributed modules will be listed with brief descriptions on the \texttt{openbecca.org} site. This list will be as exhaustive as possible, until the volume of contributions forces us to find another way to do things.

\section{How do I tell them about it?}

If you didn't record it, it didn't happen. When you do something new, make a record of it in some way so you can show everybody how cool it is. There are lots of good ways:

\begin{itemize}
\item Make a video
\item Grab screenshots
\item Draw diagrams
\item Write an appendix to this users guide\footnote{If you're new to LaTeX, just copy one of the users guide chapter files and modify it to tell your story. The \texttt{.tex} files are in the \textsc{Becca} GitHub repository under \texttt{doc/user\_manual/}.}
\item Write a conference paper\footnote{I've had good luck with the Artificial General Intelligence (AGI) and Biologically Inspired Cognitive Architectures (BICA) conference series. Also the combined International Conference on Development and Learning/Epigenetic Robotics Conference (ICDL/EpiRob) is good and the brand new AAAI Spring Symposium on Integrating Artificial Intelligence looks to be fantastic.}
\item Write a paragraph.
\end{itemize}

Then, whatever form your documentation takes, share it around. Right now, the best way to broadcast notifications to other \textsc{Becca} users is to post in the Google Group, 

\begin{verbatim}
https://groups.google.com/forum/?fromgroups#!forum/becca_users
\end{verbatim}

Sign up if you haven't yet. Incidentally, subscribing to the group's posts is also the best way to hear about others' contributions to the body of \textsc{Becca} code.

This is subject to change. The preferred way to advertise new content may eventually migrate to an \texttt{openbecca.org} forum.

