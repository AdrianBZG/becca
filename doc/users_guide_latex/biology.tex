\chapter{Biological inspiration}
 
The problem of processing and responding to a continuous high-bandwidth stream of sensor data in a complex environment has been solved by biology. The prioritizing and searching of large amounts of tactile, auditory, and visual data are performed by humans and animals all the time. This has motivated the mining of neurological structures and psychological processes (to the extent that they are understood) for hints on how to address these problems. \textsc{Becca} is not intended to be a model of how a biological brain operates, but it does rely heavily on plausibly brain-like processing to achieve its goals.

\begin{enumerate}
\item Co-activity based grouping is an expression of the Hebbian rule of thumb: "Neurons that fire together wire together." Neurons that are active simultaneously or in quick succession tend to form interconnected structures. The nature of these structures can depend critically on the relative timing of their firings, as well as how often they occur. Studies of prenatal chickens suggest that spontaneous waves of retinal activity are responsible for organizing axons in the optical pathway. As nerves grow from the retina toward the cortex, patterns of co-activity may draw the axons of nearby receptors to grow together and enable them to self-organize retinotopically.

\item {\em Hierarchy of features} is an often-hypothesized organization principle for mental representations. For example, the human visual system appears to contain an extensive hierarchy of visual features. To the extent that research has been able to determine, cortical cells in area V1 represent oriented line segments, those in V2 represent combinations of oriented lines in more complex features, and other areas (V3, V4, the mediotemporal area, etc.) represent yet more complex features.

\item {\em Lateral inhibition} is an observed behavior of neurological circuits in the cerebral cortex, resulting in winner-take-all-like function. The competition between features that share inputs is an example of behavior that might result from lateral inhibition.

\item {\em Source blindness} refers to the fact that whether a neural signal originates at a cone in the eye or a pain receptor in the toe, it has the same form by the time it reaches the brain: a pattern of spiking activity at a synapse. \textsc{Becca}'s sensor blindness was inspired by this. It has far-reaching implications. In other feature extraction methods, knowledge of the sensor type allows a certain amount of hand-crafting of first level features, such as oriented Gabor filters in the case of vision. By not making use of this information, \textsc{Becca} is forced to derive even those first level features directly from data. However, by not relying on knowledge of the sensor, \textsc{Becca} is not limited by developers' insights or the current state of neurophysiological knowledge. 

\item {\em Cerebro-cortical structure} is surprisingly uniform. This has led some to speculate about a ``common cortical algorithm'', some recurring function that all areas of the cortex perform, despite the fact that fMRI studies associate them with a vast spectrum of perceptual, behavioral, and cognitive functions. One intriguing possibility is that the cortex serves to represent features, from the lowest level up to the most abstract, that humans use to process and respond to their worlds (leaving planning and reward assignment to subcortical structures, including the basal ganglia and amygdala). If this were shown to be the case, \textsc{Becca}'s feature extractor would be one candidate for describing the operation of the cortex. It performs the same set of operations on all data, regardless of its source, and builds an arbitrarily rich hierarchy of features. This line of speculation prompted the feature extractor's development.
  
\end{enumerate}