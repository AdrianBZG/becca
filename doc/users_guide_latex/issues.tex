\chapter{Stuff that needs to be done}

There is a list of changes that will potentially improve Becca. Anyone hoping to contribute to the Becca core can use this as a suggested list of ``to do" items. Here are some of the items as of this version.

\begin{itemize}
\item{Create an extendable version of numpy arrays. There are many variables and structures in Becca that grow in size over time. A data structure that allows one- and two-dimensional numpy arrays to grow inexpensively may speed up its operation. This may be done by creating a class that uses a zero-padded numpy array that is larger than the array it is representing. As needed, it could grow in jumps of a constant factor (say, of 2) and track which portions of it are actually being used to represent the growing array. It's not space efficient, but may be much faster.}

\end{itemize}